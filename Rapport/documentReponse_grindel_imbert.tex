\documentclass[11pt,a4paper]{article}

\usepackage{classeRapport}
\usepackage{tikz}
\usepackage{algorithmeUTF8}

\setlength{\headheight}{13.59999pt}

\begin{document}

\PageDeGarde	
{rien.png} % image sur la page de garde
{Recherche opérationel - TP} % titre principal
{Balade en ville} % sous-titre
{Alexis \textsc{Imbert}\\
 Brice \textsc{Grindel}} % nom
{RO - ITI - 2022-2023} % bas de page

\Page{INSALogo}{rien.png} % logo de bas de page (en bas a droite)


\tableofcontents
\newpage
\section{Etape 1 : Modéliser, définir le problème formel, associer une classe de complexité}
\begin{itemize}
    \item 
    \begin{itemize}
        \item On propose de représenter par le graphe 
        \begin{itemize}
            \item Les noeuds représenteront les intersections, les addresses et les arrêts de métros.
            \item Les arrêtes représenteront les portions de route ou entre 2 stations de métros.
        \end{itemize}
        Si le métro est proche de d'un intersection on peut faire l'approximation que c'est le même noeud.

        Pour simplifier le graphe : on peut extraire un graphe de distance géodésique entre les adresses tel que : 
        \begin{itemize}
            \item Les noeuds représenteront les addresses
            \item Les arrêtes représenteront les chemins reliant ces addresses.
        \end{itemize}
        \item Le graphe est non orienté. Pour le passage d'addresse en paramètre on peut passer les arrêtes sur lesquels sont ces arrêtes.
        \item Représentation sagitale :
    \end{itemize}
    \item On a deux problèmes formel sous jacents.
    \begin{itemize}
        \item La recherche de plus cours chemin
        \item La recherche d'un cycle hamiltonien
    \end{itemize}
    Ce problème peut se ramener au problème du voyageur de commerce.
    \item Le problème du voyageur de commerce fait parti des problèmes NP-complet. C'est à dire que la résolution de ce type de problème est exponentiel.
    Toutefois nous ne sommes pas obliger d'abandonner tout de suite car ici le graphe est assez petit : seulement 5 addresses à parcourir.

    Le problème de plus court chemin peut etre résolu par l'algorithme de Dijkstra qui a une résolution polynomiale.
\end{itemize}
\section{Etape 2 : l'algoritme}
\paragraph{Etat de l'art}
\url{https://www.datavis.fr/playing/salesman-problem}

Une première solution est le brut force : énumération de tout les chemins possible et choix du plus cours. 
\paragraph{Algorithme}

\section{Etape 3 : L'implémentation}
\section{Etape 4 : L'adaptation}
\end{document}
